\ \\
\ \\
\label{pagsumm}
\noindent{\LARGE \sc Abstract}\\
\ \\
\ \\

\ \\

\ \\
\ \\
The automatic layout algorithms are great utility tool for the design of conceptual models, diagrams or graphs, in any modeling language like UML, ER, ORM, among others. In this sense, for achieve a layout algorithm is necessary the study of several features of that conform a correct visualized layout. We have considered of great importance, among others features, that the graph (or diagram) resulting must have a minimum crossing number between their edges. This problem is known as Crossing Number, and is NP-Complete. 

In this work we introduce the design and the implementation of {\sc ArcGen}, a new genetic algorithm, that minimize the crossing number of a graph. {\sc ArcGen} involves a preprocessing of the original graph, that transforms the graphic representation into an Arc Diagram or Linear Embedding. We described all the details of the design and the implementation of the algorithm. 

Finally, giving the motivation of the development of {\sc ArcGen} and the temporal complexity of the Crossing Number problem, some experiments are made, circumscribed to a graph with a proportional size to the diagrams generated in real practice conceptual modeling. 

We present the results of this experiments, showing that the algorithm reduces the crossing number of the original graph in up to four times. We also show the integration of the mentioned algorithm with other layout algorithms like the Force Directed Layout Algorithm by Tunkelang which allows to give a different visualization of the graph, and are evaluated and compared the overall results and by separated.

\vfill
\pagebreak
